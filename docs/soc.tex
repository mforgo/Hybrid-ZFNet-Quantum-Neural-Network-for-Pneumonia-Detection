\documentclass[12pt, a4paper]{report}

% --- ZÁKLADNÍ BALÍČKY ---
\usepackage[utf8]{inputenc}
\usepackage[T1]{fontenc}
\usepackage[czech]{babel}
\usepackage{graphicx}       % Pro vkládání obrázků (logo)
\usepackage{setspace}       % Pro řádkování 1.5
\usepackage{geometry}       % Nastavení okrajů
\usepackage{mathptmx}       % Písmo podobné Times New Roman
\usepackage{titlesec}       % Úprava nadpisů
\usepackage{csquotes}       % Pro správné uvozovky
\usepackage[hidelinks]{hyperref} % Proklikávací obsah bez rámečků

% --- NASTAVENÍ CITACÍ (ISO 690) ---
% Pokud používáte Overleaf, ujistěte se, že máte nastavený kompiler na pdflatex nebo xelatex
\usepackage[style=iso-numeric, backend=biber]{biblatex}
\addbibresource{literatura.bib} % Vytvořte soubor literatura.bib

% --- NASTAVENÍ ROZMĚRŮ DLE ŠABLONY ---
\geometry{
 a4paper,
 left=25mm,
 right=25mm,
 top=25mm,
 bottom=25mm,
}
\onehalfspacing % Řádkování 1.5 

% --- DEFINICE ÚDAJŮ O PRÁCI ---
\newcommand{\nazevPraceCZ}{Hybridní model pro detekci pneumonie}
\newcommand{\nazevPraceEN}{Hybrid model for pneumonia detection}
\newcommand{\autor}{Michal Forgó}
\newcommand{\skola}{Střední škola informatiky a finančních služeb, Klatovská 200 G, 301 00, Plzeň}
\newcommand{\kraj}{Plzeňský kraj}
\newcommand{\konzultant}{Bc. Jan Boháč}
\newcommand{\obor}{Matematika a data science (1)}
\newcommand{\rok}{2025}
\newcommand{\misto}{Plzeň}

\begin{document}

% ================= TITULNÍ STRANA [cite: 1-9] =================
\begin{titlepage}
    \begin{center}
        \Large \textbf{STŘEDOŠKOLSKÁ ODBORNÁ ČINNOST} \\
        \vspace{1cm}
        
        % Místo pro logo - nahrajte soubor logo_soc.png nebo smažte
        % \includegraphics[width=4cm]{logo_soc.png} 
        \vspace{2cm}

        \Large Obor: \obor \\
        \vspace{2cm}

        \Huge \textbf{\nazevPraceCZ} \\
        \vspace{0.5cm}
        \Large \textit{\nazevPraceEN} \\
        
        \vfill
    \end{center}

    \begin{flushleft}
        \textbf{Autor:} \autor \\
        \textbf{Škola:} \skola \\
        \textbf{Kraj:} \kraj \\
        \textbf{Konzultant:} \konzultant
    \end{flushleft}

    \begin{center}
        \vspace{1cm}
        \misto \ \rok
    \end{center}
\end{titlepage}

% ================= PROHLÁŠENÍ [cite: 10-34] =================
\newpage
\thispagestyle{empty}
\section*{Prohlášení}

Prohlašuji, že jsem svou práci SOČ vypracoval/a samostatně a použil/a jsem pouze prameny a literaturu uvedené v seznamu bibliografických záznamů.

Beru na vědomí, že nejpozději odevzdáním slovesné vědecké práce do veřejné soutěže Středoškolská odborná činnost, stejně jako odevzdáním jejích příloh a dalších připojených děl, např. audiovizuálních, fotografických, výtvarných, architektonických apod. (dále jen „soutěžní dílo“), dochází ke zveřejnění díla podle § 4 odst. 1 zákona č. 121/2000 Sb., autorského zákona, ve znění pozdějších předpisů (dále jen „autorský zákon“).

Totéž platí pro pozdější odevzdání doplněného, změněného, upraveného nebo opraveného díla.

Beru na vědomí, že zveřejněním díla, jehož součástí je vynález, se tento vynález stává součástí stavu techniky podle § 5 odst. 1, 2 zákona č. 527/1990 Sb., o vynálezech, průmyslových vzorech a zlepšovacích návrzích, ve znění pozdějších předpisů (dále jen „patentový zákon“), což zakládá překážku pro udělení patentu podle § 3 odst. 1 patentového zákona.

Beru na vědomí, že vyhlašovatel soutěže je podle § 61 odst. 1 autorského zákona per analogiam oprávněn užít soutěžní dílo pro účely zajištění průběhu soutěže, zejména k zajištění transparentnosti soutěže a veřejnosti obhajob soutěžních prací. V odůvodněném rozsahu je tedy vyhlašovatel po dobu účasti autora v soutěži oprávněn zejména:
\begin{itemize}
    \item zhotovovat rozmnoženiny díla, je-li to nezbytné k seznámení účastníků soutěže, porotců nebo veřejnosti se soutěžní prací;
    \item zapůjčit originál nebo rozmnoženinu díla účastníkům soutěže, porotcům nebo veřejnosti. Přitom dbá na bezpečné nakládání s dílem;
    \item vystavovat originál nebo rozmnoženinu díla v průběhu soutěžních přehlídek a doprovodných akcí;
    \item sdělovat dílo veřejnosti v nehmotné podobě, a to především počítačovou nebo obdobnou sítí.
\end{itemize}

% --- VOLBA AI PROHLÁŠENÍ (ODKOMENTUJTE JEN JEDNU MOŽNOST) ---

% MOŽNOST A: BEZ AI [cite: 26]
% Dále prohlašuji, že při tvorbě této práce jsem nepoužil/a nástroje AI.

% MOŽNOST B: S POUŽITÍM AI [cite: 28-29]
Dále prohlašuji, že při tvorbě této práce jsem použil nástroj generativního modelu AI [NÁZEV APLIKACE; WEBOVÁ ADRESA APLIKACE] za účelem [DŮVOD]. Po použití tohoto nástroje jsem provedl/a kontrolu obsahu a přebírám za něj plnou zodpovědnost.

\vspace{2cm}

V \misto \ dne \today \hfill .......................................................\\
\hspace*{10cm} \autor

% ================= PODĚKOVÁNÍ [cite: 35-38] =================
\newpage
\thispagestyle{empty}
\section*{Poděkování}
Rád bych na tomto místě poděkoval Bc. Janu Boháčovi za odborné vedení a cenné rady, které mi poskytl při zpracování této práce.

Dále mé poděkování patří výzkumnému centru Nové technologie (NTC) Západočeské univerzity v Plzni za poskytnutí přístupu ke kvantovým počítačům IBM a možnost jejich využití při řešení této práce. Jmenovitě bych chtěl v této souvislosti poděkovat Ing. Vítu Nováčkovi, Ph.D.

% ================= ANOTACE A KLÍČOVÁ SLOVA [cite: 39-48] =================
\newpage
\thispagestyle{empty}

\section*{Anotace}
Tato práce se zabývá využitím kvantového strojového učení v oblasti medicínské diagnostiky, konkrétně při detekci zápalu plic z rentgenových snímků hrudníku. Hlavním cílem bylo navrhnout a implementovat hybridní model sítě, který kombinuje klasickou CNN s kvantovými výpočty. V rámci experimentální části byla CNN využita jako extraktor příznaků (feature extractor), jehož výstup byl následně zpracován variačním kvantovým obvodem (VQC) sloužícím jako klasifikátor. Účinnost tohoto hybridního přístupu byla testována na veřejně dostupné databázi rentgenových snímků a porovnána s výkonem konvenčních metod hlubokého učení. Výsledky práce demonstrují, zda zapojení kvantových vrstev přináší zlepšení v přesnosti či efektivitě učení oproti čistě klasickým modelům. Práce tak přispívá k diskusi o praktické využitelnosti kvantových neuronových sítí v analýze biomedicínských obrazových dat.

\section*{Klíčová slova}
strojové učení; kvantové počítání; detekce pneumonie; NISQ; hybridní model

\vspace{1cm}
\hrule
\vspace{1cm}

\section*{Annotation}
This thesis deals with the application of quantum machine learning in medical diagnostics, specifically in the detection of pneumonia from chest X-ray images. The main objective was to design and implement a hybrid network model combining a classical CNN with quantum computing. In the experimental part, a CNN was utilized as a feature extractor, the output of which was subsequently processed by a Variational Quantum Circuit (VQC) serving as a classifier. The efficacy of this hybrid approach was tested on a publicly available dataset of X-ray images and compared with the performance of conventional deep learning methods. The results demonstrate whether the integration of quantum layers yields improvements in accuracy or learning efficiency compared to purely classical models. The thesis thus contributes to the discussion on the practical applicability of quantum neural networks in the analysis of biomedical image data.

\section*{Keywords}
machine learning; quantum computing; pneumonia detection; NISQ; hybrid model

% ================= OBSAH [cite: 49] =================
\newpage
\tableofcontents

% ================= VLASTNÍ TEXT PRÁCE =================
\newpage
% Číslování stránek začíná být viditelné zde, ale počítá se od začátku
\setcounter{page}{7} % Dle šablony obsah končí cca na str. 6 [cite: 50]

\chapter{Úvod} \label{chap:uvod}

Rychlý rozvoj umělé inteligence a hlubokého učení v posledním desetiletí zásadním způsobem transformoval řadu oborů, přičemž medicínská diagnostika patří mezi ty nejvíce ovlivněné. Konvoluční neuronové sítě (CNN) dnes dosahují při analýze rentgenových snímků přesnosti srovnatelné s lidskými experty. Přesto naráží klasické křemíkové čipy na své fyzikální limity, zejména pokud jde o výpočetní náročnost trénování stále komplexnějších modelů. Do popředí zájmu se tak dostává kvantové počítání, které slibuje revoluci ve způsobu zpracování informací.

V současné době se oblast kvantových technologií nachází v éře, kterou John Preskill definoval jako NISQ (Noisy Intermediate-Scale Quantum). Jedná se o období, kdy máme k dispozici kvantové procesory s desítkami až stovkami fyzických qubitů. Tyto procesory sice již nejsou triviální, ale zároveň ještě nejsou dostatečně robustní a chybově korigované (fault-tolerant), aby mohly provádět libovolně dlouhé algoritmy bez vlivu šumu a dekoherence.

I v tomto nedokonalém prostředí se již podařilo na reálném hardwaru provést výpočty, které naznačují dosažení tzv. kvantové výhody či dokonce kvantové nadvlády (quantum supremacy). Experimenty společností jako Google či IBM ukázaly, že kvantové čipy dokáží vyřešit specifické matematické úlohy řádově rychleji než nejvýkonnější klasické superpočítače. Je však nutné podotknout, že tato tvrzení jsou předmětem vědeckých diskusí a kontroverzí, neboť klasické algoritmy jsou neustále optimalizovány a hranice mezi tím, co je a není klasicky simulovatelné, se dynamicky posouvá.

Zásadní otázkou, kterou si tato práce klade, je aplikovatelnost těchto principů v oblasti strojového učení (Quantum Machine Learning - QML). Je sporné, zda v éře NISQ, která je charakteristická vysokou chybovostí a nedostatkem logických qubitů, můžeme očekávat reálné zlepšení oproti klasickým neuronovým sítím. Rušení kvantových stavů může v mnoha případech zcela degradovat teoretickou výhodu, kterou kvantový paralelizmus nabízí.

Tato práce se pokouší na tuto otázku odpovědět prostřednictvím návrhu hybridní architektury. Namísto snahy o čistě kvantové řešení, které by v současnosti naráželo na hardwarové limity, volíme kombinovaný přístup. Spojujeme osvědčenou klasickou architekturu ZFNet, která slouží k efektivní extrakci příznaků z medicínských snímků, s variačním kvantovým obvodem (VQC). Cílem je zjistit, zda i malé množství "šumících" qubitů zapojených do rozhodovacího procesu může přinést měřitelnou výhodu v přesnosti detekce pneumonie, nebo zda je vliv šumu v současné generaci hardwaru stále příliš dominantní.

\chapter{Teoretická část} \label{chap:teorie}
% [cite: 98-109]
\section{Medicínská východiska} \label{sec:medicina}

Pneumonie, běžně označovaná jako zápal plic, představuje jedno z nejrozšířenějších a nejzávažnějších respiračních onemocnění celosvětově. Jedná se o akutní zánětlivý proces postihující plicní parenchym, konkrétně plicní sklípky (alveoly) a terminální bronchioly, které se v důsledku zánětu plní tekutinou a hnisem. Tento stav výrazně omezuje schopnost plic absorbovat kyslík, což vede k respirační nedostatečnosti \cite{who_pneumonia}. Podle statistik Světové zdravotnické organizace (WHO) patří pneumonie mezi hlavní příčiny úmrtí dětí do pěti let a představuje významné riziko pro seniory a imunokompromitované pacienty.

\subsection{Etiologie a klasifikace}
Původci onemocnění mohou být různého charakteru, přičemž správná identifikace patogenu je klíčová pro zvolení adekvátní léčby. Nejčastěji se setkáváme s pneumonií:
\begin{itemize}
    \item \textbf{Bakteriální:} Nejběžnějším původcem je \textit{Streptococcus pneumoniae}. Tento typ se často projevuje náhlým nástupem a na rentgenových snímcích mívá charakteristický obraz lobární konsolidace (postižení celého laloku).
    \item \textbf{Virovou:} Způsobenou viry chřipky (Influenza), RSV nebo koronaviry (např. SARS-CoV-2). Virové pneumonie mají tendenci vykazovat difuznější nález a postihovat interstitium (vazivovou tkáň plic) \cite{radiology_viral_vs_bacterial}.
    \item \textbf{Atypickou a mykotickou:} Způsobenou méně běžnými organismy, jako jsou mykoplazmata nebo houby.
\end{itemize}

\subsection{Diagnostika pomocí skiagrafie hrudníku}
Ačkoliv je výpočetní tomografie (CT) považována za citlivější metodu pro detailní zobrazení plicních patologií, základním pilířem diagnostiky zůstává konvenční skiagrafie hrudníku (rentgen, CXR). Důvody jsou pragmatické: rentgenové vyšetření je rychlé, levné, široce dostupné a vystavuje pacienta výrazně nižší dávce ionizujícího záření než CT \cite{chest_xray_diagnosis}.

Na rentgenovém snímku se zdravá plicní tkáň, která je naplněná vzduchem, jeví jako tmavá oblast (radiolucentní). Naopak patologické procesy spojené s pneumonií, jako jsou zánětlivé výpotky a buněčná infiltrace, absorbují rentgenové záření více, a proto se na snímku zobrazují jako světlá až bílá zastínění (opacity).

Lékař-radiolog při popisu snímku hledá specifické příznaky:
\begin{itemize}
    \item \textbf{Konsolidace:} Oblast plic, kde byl vzduch nahrazen tekutinou, jevící se jako jasná bílá skvrna.
    \item \textbf{Vzdušný bronchogram:} Viditelné průdušky procházející zkonsolidovanou tkání.
    \item \textbf{Infiltráty:} Méně ohraničená zastínění, která mohou být skvrnitá nebo difuzní.
\end{itemize}

\subsection{Limitace manuální diagnostiky a role AI}
Interpretace rentgenových snímků je vysoce náročný proces, který vyžaduje zkušeného radiologa. I přesto je tento proces zatížen určitou mírou subjektivity. Studie ukazují, že shoda mezi různými radiology (inter-observer variability) není stoprocentní, zejména u hraničních nálezů nebo u snímků s nízkým kontrastem \cite{radiologist_error_rate}.

Mezi hlavní problémy manuální diagnostiky patří:
\begin{enumerate}
    \item \textbf{Vizuální podobnost patologií:} Obraz pneumonie může být snadno zaměnitelný s jinými stavy, jako je edém plic, atelektáza nebo plicní fibróza.
    \item \textbf{Lidský faktor:} Únava lékařů při vyhodnocování velkého množství snímků (např. během epidemií) zvyšuje riziko přehlédnutí nálezu (falešná negativita).
    \item \textbf{Dostupnost expertů:} V rozvojových zemích nebo odlehlých oblastech může být nedostatek kvalifikovaných radiologů kritický.
\end{enumerate}

Právě tyto limitace otevírají prostor pro aplikaci systémů počítačového vidění a umělé inteligence. Algoritmy, jako jsou konvoluční neuronové sítě (CNN), a v kontextu této práce i hybridní kvantové sítě, mají potenciál sloužit jako "druhý pár očí", který dokáže konzistentně detekovat jemné vzory v obraze, jež mohou lidskému oku uniknout.



\chapter{Cíle práce} \label{chap:cile}
% [cite: 110-114]
Definujte hlavní a dílčí cíle. Stanovte hypotézy nebo výzkumné otázky.

\chapter{Metodika} \label{chap:metodika}
% [cite: 115-122]
Popis materiálu, dat a metod. Musí být reprodukovatelné.
\begin{equation}
    E = mc^2
\end{equation}
Znaky veličin se zapisují kurzívou (např. $T$, $t$), jednotky s mezerou ($T = 90$ °C) [cite: 93-94].

\chapter{Výsledky} \label{chap:vysledky}
% [cite: 123-135]
Prezentace vlastních výsledků, grafů a tabulek.

\chapter{Diskuze} \label{chap:diskuze}
% [cite: 136-146]
Interpretace výsledků, srovnání s literaturou uvedenou v teoretické části. Limity práce.

\chapter{Závěr} \label{chap:zaver}
% [cite: 147-150]
Stručné shrnutí, naplnění cílů, přínos práce. Rozsah cca 1 strana.

% ================= LITERATURA [cite: 151-190] =================
\printbibliography[title={Seznam použité literatury}]
% Poznámka: SOČ preferuje ISO 690.

% ================= SEZNAM ZKRATEK [cite: 191-197] =================
\chapter*{Seznam použitých zkratek}
\addcontentsline{toc}{chapter}{Seznam použitých zkratek}
\begin{description}
    \item[AI] Umělá inteligence
    \item[SOČ] Středoškolská odborná činnost
    \item[NISQ] Noisy Intermediate-Scale Quantum
    \item[CNN] Convolutional Neural Network
    \item[VQC] Variační kvantový klasifikátor
    \item[QML] Quantum Machine Learning
\end{description}

% ================= PŘÍLOHY [cite: 198-202] =================
\appendix
\chapter{Název přílohy 1}
Zde vložte rozměrné tabulky, grafy nebo fotodokumentaci. Přílohy se číslují průběžně.

\end{document}